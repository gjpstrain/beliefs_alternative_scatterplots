\documentclass[manuscript,screen,review]{acmart}


\IfFileExists{upquote.sty}{\usepackage{upquote}}{}
\IfFileExists{microtype.sty}{% use microtype if available
  \usepackage[]{microtype}
  \UseMicrotypeSet[protrusion]{basicmath} % disable protrusion for tt fonts
}{}
\makeatletter
\@ifundefined{KOMAClassName}{% if non-KOMA class
  \IfFileExists{parskip.sty}{%
    \usepackage{parskip}
  }{% else
    \setlength{\parindent}{0pt}
    \setlength{\parskip}{6pt plus 2pt minus 1pt}}
}{% if KOMA class
  \KOMAoptions{parskip=half}}
\makeatother

%%
%% This is file `sample-manuscript.tex',
%% generated with the docstrip utility.
%%
%% The original source files were:
%%
%% samples.dtx  (with options: `manuscript')
%% 
%% IMPORTANT NOTICE:
%% 
%% For the copyright see the source file.
%% 
%% Any modified versions of this file must be renamed
%% with new filenames distinct from sample-manuscript.tex.
%% 
%% For distribution of the original source see the terms
%% for copying and modification in the file samples.dtx.
%% 
%% This generated file may be distributed as long as the
%% original source files, as listed above, are part of the
%% same distribution. (The sources need not necessarily be
%% in the same archive or directory.)
%%
%%
%% Commands for TeXCount
%TC:macro \cite [option:text,text]
%TC:macro \citep [option:text,text]
%TC:macro \citet [option:text,text]
%TC:envir table 0 1
%TC:envir table* 0 1
%TC:envir tabular [ignore] word
%TC:envir displaymath 0 word
%TC:envir math 0 word
%TC:envir comment 0 0
%%
%%
%% The first command in your LaTeX source must be the \documentclass command.


% Options for packages loaded elsewhere
\PassOptionsToPackage{unicode}{hyperref}
\PassOptionsToPackage{hyphens}{url}
\PassOptionsToPackage{dvipsnames,svgnames,x11names}{xcolor}

\IfFileExists{bookmark.sty}{\usepackage{bookmark}}{\usepackage{hyperref}}

%% PANDOC PREAMBLE BEGINS


\providecommand{\tightlist}{%
  \setlength{\itemsep}{0pt}\setlength{\parskip}{0pt}}\usepackage{longtable,booktabs,array}
\usepackage{calc} % for calculating minipage widths
% Correct order of tables after \paragraph or \subparagraph
\usepackage{etoolbox}
\makeatletter
\patchcmd\longtable{\par}{\if@noskipsec\mbox{}\fi\par}{}{}
\makeatother
% Allow footnotes in longtable head/foot
\IfFileExists{footnotehyper.sty}{\usepackage{footnotehyper}}{\usepackage{footnote}}
\makesavenoteenv{longtable}
\usepackage{graphicx}
\makeatletter
\def\maxwidth{\ifdim\Gin@nat@width>\linewidth\linewidth\else\Gin@nat@width\fi}
\def\maxheight{\ifdim\Gin@nat@height>\textheight\textheight\else\Gin@nat@height\fi}
\makeatother
% Scale images if necessary, so that they will not overflow the page
% margins by default, and it is still possible to overwrite the defaults
% using explicit options in \includegraphics[width, height, ...]{}
\setkeys{Gin}{width=\maxwidth,height=\maxheight,keepaspectratio}
% Set default figure placement to htbp
\makeatletter
\def\fps@figure{htbp}
\makeatother

\usepackage{booktabs}
\usepackage{longtable}
\usepackage{array}
\usepackage{multirow}
\usepackage{wrapfig}
\usepackage{float}
\usepackage{colortbl}
\usepackage{pdflscape}
\usepackage{tabu}
\usepackage{threeparttable}
\usepackage{threeparttablex}
\usepackage[normalem]{ulem}
\usepackage{makecell}
\usepackage{xcolor}
\definecolor{mypink}{RGB}{219, 48, 122}
\makeatletter
\@ifpackageloaded{caption}{}{\usepackage{caption}}
\AtBeginDocument{%
\ifdefined\contentsname
  \renewcommand*\contentsname{Table of contents}
\else
  \newcommand\contentsname{Table of contents}
\fi
\ifdefined\listfigurename
  \renewcommand*\listfigurename{List of Figures}
\else
  \newcommand\listfigurename{List of Figures}
\fi
\ifdefined\listtablename
  \renewcommand*\listtablename{List of Tables}
\else
  \newcommand\listtablename{List of Tables}
\fi
\ifdefined\figurename
  \renewcommand*\figurename{Figure}
\else
  \newcommand\figurename{Figure}
\fi
\ifdefined\tablename
  \renewcommand*\tablename{Table}
\else
  \newcommand\tablename{Table}
\fi
}
\@ifpackageloaded{float}{}{\usepackage{float}}
\floatstyle{ruled}
\@ifundefined{c@chapter}{\newfloat{codelisting}{h}{lop}}{\newfloat{codelisting}{h}{lop}[chapter]}
\floatname{codelisting}{Listing}
\newcommand*\listoflistings{\listof{codelisting}{List of Listings}}
\makeatother
\makeatletter
\makeatother
\makeatletter
\@ifpackageloaded{caption}{}{\usepackage{caption}}
\@ifpackageloaded{subcaption}{}{\usepackage{subcaption}}
\makeatother
%% PANDOC PREAMBLE ENDS

\setlength{\parindent}{10pt}
\setlength{\parskip}{0pt}

\hypersetup{
  pdftitle={Changing Beliefs About Correlations in Atypical Scatterplots},
  pdfauthor={Gabriel Strain; Andrew J. Stewart; Paul Warren; Charlotte Rutherford; Caroline Jay},
  colorlinks=true,
  linkcolor={blue},
  filecolor={Maroon},
  citecolor={Blue},
  urlcolor={red},
  pdfcreator={LaTeX via pandoc, via quarto}}

%% \BibTeX command to typeset BibTeX logo in the docs
\AtBeginDocument{%
  \providecommand\BibTeX{{%
    Bib\TeX}}}

%% Rights management information.  This information is sent to you
%% when you complete the rights form.  These commands have SAMPLE
%% values in them; it is your responsibility as an author to replace
%% the commands and values with those provided to you when you
%% complete the rights form.
\setcopyright{acmcopyright}
\copyrightyear{2018}
\acmYear{2018}
\acmDOI{XXXXXXX.XXXXXXX}

%% These commands are for a PROCEEDINGS abstract or paper.
\acmConference[Conference acronym 'XX]{Make sure to enter the correct
conference title from your rights confirmation emai}{June 03--05,
2018}{Woodstock, NY}
\acmPrice{15.00}
\acmISBN{978-1-4503-XXXX-X/18/06}

%% Submission ID.
%% Use this when submitting an article to a sponsored event. You'll
%% receive a unique submission ID from the organizers
%% of the event, and this ID should be used as the parameter to this command.
%%\acmSubmissionID{123-A56-BU3}

%%
%% For managing citations, it is recommended to use bibliography
%% files in BibTeX format.
%%
%% You can then either use BibTeX with the ACM-Reference-Format style,
%% or BibLaTeX with the acmnumeric or acmauthoryear sytles, that include
%% support for advanced citation of software artefact from the
%% biblatex-software package, also separately available on CTAN.
%%
%% Look at the sample-*-biblatex.tex files for templates showcasing
%% the biblatex styles.
%%

%%
%% The majority of ACM publications use numbered citations and
%% references.  The command \citestyle{authoryear} switches to the
%% "author year" style.
%%
%% If you are preparing content for an event
%% sponsored by ACM SIGGRAPH, you must use the "author year" style of
%% citations and references.
%% Uncommenting
%% the next command will enable that style.
%%\citestyle{acmauthoryear}


%% end of the preamble, start of the body of the document source.
\begin{document}


%%
%% The "title" command has an optional parameter,
%% allowing the author to define a "short title" to be used in page headers.
\title{Changing Beliefs About Correlations in Atypical Scatterplots}

%%
%% The "author" command and its associated commands are used to define
%% the authors and their affiliations.
%% Of note is the shared affiliation of the first two authors, and the
%% "authornote" and "authornotemark" commands
%% used to denote shared contribution to the research.


  \author{Gabriel Strain}
  \orcid{0000-0002-4769-9221}
            \affiliation{%
                  \institution{Department of Computer Science, Faculty
of Science and Engineering, University of Manchester}
                          \streetaddress{Oxford Road}
                          \city{Manchester}
                                  \country{United Kingdom}
                          \postcode{M13 9PL}
              }
        \author{Andrew J. Stewart}
  
            \affiliation{%
                  \institution{Department of Computer Science, Faculty
of Science and Engineering, University of Manchester}
                          \streetaddress{Oxford Road}
                          \city{Manchester}
                                  \country{United Kingdom}
                          \postcode{M13 9PL}
              }
        \author{Paul Warren}
  
            \affiliation{%
                  \institution{Division of Psychology, Communication and
Human Neuroscience, School of Health Sciences, Faculty of Biology,
Medicine, and Health, University of Manchester}
                          \streetaddress{Oxford Road}
                          \city{Manchester}
                                  \country{United Kingdom}
                          \postcode{M13 9PL}
              }
        \author{Charlotte Rutherford}
  
            \affiliation{%
                  \institution{Division of Psychology Communication and
Human Neuroscience, School of Health Sciences, Faculty of Biology,
Medicine, and Health, University of Manchester}
                          \streetaddress{Oxford Road}
                          \city{Manchester}
                                  \country{United Kingdom}
                          \postcode{M13 9PL}
              }
        \author{Caroline Jay}
  
            \affiliation{%
                  \institution{Department of Computer Science, Faculty
of Science and Engineering, University of Manchester}
                          \streetaddress{Oxford Road}
                          \city{Manchester}
                                  \country{United Kingdom}
                          \postcode{M13 9PL}
              }
      
\renewcommand{\shortauthors}{Strain et al.}

%% By default, the full list of authors will be used in the page
%% headers. Often, this list is too long, and will overlap
%% other information printed in the page headers. This command allows
%% the author to define a more concise list
%% of authors' names for this purpose.
%\renewcommand{\shortauthors}{Trovato et al.}
%%  
%% The abstract is a short summary of the work to be presented in the
%% article.
\begin{abstract}
abstract goes here    
\end{abstract}

%%
%% The code below is generated by the tool at http://dl.acm.org/ccs.cfm.
%% Please copy and paste the code instead of the example below.
%%
\begin{CCSXML}
<ccs2012>
 <concept>
  <concept_id>10010520.10010553.10010562</concept_id>
  <concept_desc>Computer systems organization~Embedded systems</concept_desc>
  <concept_significance>500</concept_significance>
 </concept>
 <concept>
  <concept_id>10010520.10010575.10010755</concept_id>
  <concept_desc>Computer systems organization~Redundancy</concept_desc>
  <concept_significance>300</concept_significance>
 </concept>
 <concept>
  <concept_id>10010520.10010553.10010554</concept_id>
  <concept_desc>Computer systems organization~Robotics</concept_desc>
  <concept_significance>100</concept_significance>
 </concept>
 <concept>
  <concept_id>10003033.10003083.10003095</concept_id>
  <concept_desc>Networks~Network reliability</concept_desc>
  <concept_significance>100</concept_significance>
 </concept>
</ccs2012>
\end{CCSXML}

\ccsdesc[500]{Computer systems organization~Embedded systems}
\ccsdesc[300]{Computer systems organization~Redundancy}
\ccsdesc{Computer systems organization~Robotics}
\ccsdesc[100]{Networks~Network reliability}

%%
%% Keywords. The author(s) should pick words that accurately describe
%% the work being presented. Separate the keywords with commas.
\keywords{belief change, correlation
perception, scatterplot, crowdsourced}


%%
%% This command processes the author and affiliation and title
%% information and builds the first part of the formatted document.
\maketitle

\setlength{\parskip}{-0.1pt}

\section{Introduction}\label{sec-intro-main}

\section{Related Work}\label{sec-rel-work-main}

\section{General Methods}\label{sec-general-methods}

In this section we discuss our general research methods, including our
implementations of open research practices, our approach to and
justification for crowdsourcing, and our approach to stimulus
generation.

\subsection{Open Research}\label{sec-open-research}

Both our pre and main studies were conducted according to the principles
of open and reproducible research \citep{ayris_2018}. We pre-registered
hypotheses and analysis plans with the Open Science Framework (OSF) for
the pre-study\footnote{https://osf.io/xuf4d} and the main
experiment\footnote{tbc}, and there were no deviations from them. All
data and analysis code are included in a GitHub repository\footnote{https://github.com/gjpstrain/beliefs\_attitudes\_atypical}.
This repository contains instructions for building a Docker container
\citep{merkel_2014} that reproduces the computational environment the
paper was written in. This allows for full replication of stimuli,
figures, analysis, and the paper itself. Ethical approval was granted by
the (removed for anon).

\subsection{Crowdsourcing}\label{crowdsourcing}

While much prior work into correlation perception in scatterplots has
taken place in person, there is precedence for work that explores
cognition to take place online using crowdsourced participants
\citep{xiong_2022}. Crowdsourcing not only affords us recruitment of
samples from across our lay population of interest, it is considerably
quicker and less expensive than in-person testing. We therefore choose
to crowdsource all participants. Previous work has reported issues of
data quality and skewed demographics
\citep{chmielewski_2020, charalambides_2021, peer_2021}, so we follow
published guidelines \citep{peer_2021} to give us the best chance of
collecting high quality data. We use the Prolific.co platform
\citep{prolific} with strict pre-screening criteria; participants were
required to have completed at least 100 studies using Prolific, and were
required to have a Prolific score of 100, representing a 99\% approval
rate.

\section{Pre-Study: Investigating Beliefs About Relatedness
Statements}\label{sec-pre-study}

\subsection{Introduction}\label{sec-pre-study-intro}

\subsubsection{Testing Beliefs}\label{sec-testing-beliefs}

\subsubsection{Preparation of Stimuli}\label{sec-stim-prep-pre}

Due to previous evidence suggesting effects of prior belief strength and
topic emotionality on the propensity for belief change, we first aim to
build a picture of people's thoughts and feelings along these dimensions
in our population of interest. With the intention of testing the
potential for changes in beliefs about correlations displayed in
scatterplots depicting weak and strong correlations, and those whose
topics were both strong and neutral in emotional valence, we began by
using ChatGPT4 \citep{chatgpt} to generate 100 correlation statements
using the following prompt:

\begin{center}

    ``Generate 100 statements that describe the correlation between two variables, such as :

     "X is associated with a higher level of Y" or

     "As X increases, Y increases".

    Try to match all the statements on emotionality.``
    
\end{center}

The full list of these statements can be found in the supplementary
materials. Note that we cite our use of ChatGPT according to the AI Code
of Conduct developed by Iliada Eleftheriou and Ajmal Mubarik and the
University of Manchester \citep{iliada_2023}. Two authors rated each
statement on topic emotionality and strength of correlation using Likert
scales from 1 to 7. Topic emotionality had a midpoint at 4, whereas
strength of correlation varied between 1 (Not Related At All) and 7
(Strongly Related). We calculated a quadratic weighted Cohen's Kappa
between the two raters using the \textbf{irr} package (version 0.84.1
\citep{irr}), in order to penalise larger magnitude disagreements more
harshly. We found agreement above chance for both topic emotionality
(\(\kappa\) = 0.49, \emph{p} \textless{} .001) and strength of
correlation (\(\kappa\) = 0.51, \emph{p} \textless{} .001), indicating
moderate levels of agreement in both cases
\citep{cohen_1968, fleiss_1969}.

\begin{table}

\caption{\label{tbl-pre-test-hi}Pre-test statements that were rated as
being strongly correlated.}

\centering{

\begin{tabular}[t]{rl}
\toprule
Item Number & Statement - Strong Correlation Depicted\\
\midrule
1 & Increased exposure to sunlight is correlated with higher vitamin D levels.\\
2 & As caffeine consumption increases, so does the average heart rate.\\
3 & Greater frequency of exercise is linked to a lower risk of depression.\\
4 & Greater use of helmets is associated with a lower incidence of head injuries in cyclists.\\
5 & As the quality of healthcare improves, life expectancy tends to increase.\\
\addlinespace
6 & As access to clean water improves, the incidence of waterborne diseases decreases.\\
7 & Higher levels of empathy are linked to stronger interpersonal relationships.\\
8 & As soil quality degrades, agricultural productivity tends to decrease.\\
9 & Higher levels of civic engagement are linked to a stronger sense of community.\\
10 & Higher sugar consumption is associated with an increased risk of dental cavities.\\
\addlinespace
11 & Higher attendance at preventive health screenings is linked to earlier detection of diseases.\\
12 & Increased use of energy-efficient appliances is associated with lower electricity bills.\\
13 & As pedestrian-friendly infrastructure improves, urban walkability tends to increase.\\
14 & Greater regularity in sleep patterns is associated with improved mental health.\\
\bottomrule
\end{tabular}

}

\end{table}%

\begin{table}

\caption{\label{tbl-pre-test-low}Pre-test statements that were rated as
being weakly correlated.}

\centering{

\begin{tabular}[t]{rl}
\toprule
Item Number & Statement - Weak Correlation Depicted\\
\midrule
15 & Greater water consumption is linked to improved kidney function.\\
16 & As the amount of sleep decreases, the risk of obesity increases.\\
17 & Greater intake of omega-3 fatty acids is associated with lower inflammation levels.\\
18 & Greater exposure to music education is linked to enhanced cognitive development in children.\\
19 & Higher exposure to air conditioning is associated with increased respiratory issues.\\
\addlinespace
20 & Higher frequency of family meals is linked to better eating habits in children.\\
21 & As participation in community arts programs increases, local cultural engagement tends to rise.\\
22 & Higher consumption of spicy foods is associated with a lower risk of certain types of cancer.\\
23 & Greater adherence to a Mediterranean diet is linked to a lower risk of neurodegenerative diseases.\\
24 & Higher consumption of nuts and seeds is associated with reduced risk of cardiovascular diseases.\\
\addlinespace
25 & As cultural preservation efforts increase, community identity and cohesion tend to strengthen.\\
\bottomrule
\end{tabular}

}

\end{table}%

Following this, we selected strongly and weakly correlated statements
with the highest level of absolute agreement, resulting in the 14
strongly correlated statements that can be seen in
Table~\ref{tbl-pre-test-hi} and the 11 weakly correlated statements that
can be seen in Table~\ref{tbl-pre-test-low}. We then tested these 25
statements with a representative UK sample in order to ascertain
consensus on both topic emotionality and strength of correlation. Doing
so allows us to effectively exclude these factors when we analyse the
effects of our atypical scatterplot designs on the propensity for belief
change in our main experiment.

\subsection{Method}\label{sec-method-pre}

\subsubsection{Participants}\label{sec-participants-pre}

100 participants were recruited using the Prolific.co platform
\citep{prolific}. English fluency and residency was required for
participation, as our main experiment relied on familiarity with data
visualisations from a popular British news source. In addition to 25
experimental items, we included six attention check items, which asked
participants to provide specific answers. No participants failed more
than 2 out of 6 attention check items, and therefore data from all 100
were included in the full analysis (52.0\% male and 48.0\% female.
Participants' mean age was 41.1 (\emph{SD} = 12.3). The average time
taken to complete the survey was 7.6 minutes (\emph{SD} = 2.9 minutes).

\subsubsection{Design}\label{sec-design-pre}

Each participant saw all survey items (Table~\ref{tbl-pre-test-hi} and
Table~\ref{tbl-pre-test-low}), along with the six attention check items,
in a fully randomised order. All experimental code, materials, and
instructions are hosted on GitLab\footnote{https://gitlab.pavlovia.org/Strain/beliefs\_scatterplots\_pretest}.

\subsubsection{Procedure}\label{sec-procedure-pre}

The experiment was built using Psychopy \citep{pierce_2019} and hosted
on Pavlovia.org. Participants were permitted to complete the experiment
using a phone, tablet, desktop, or laptop computer. Participants were
first shown the participant information sheet and were asked to provide
consent through key presses in response to consent statements. They were
asked to provide their age in a free text box, followed by their gender
identity. Participants were told that they would be asked to read
statements about the relatedness between a pair of variables, after
which they would have to indicate their beliefs about topic emotionality
and the strength of correlation suggested using a pair of sliders. To
familiarize themselves with the sliders, they were asked to complete a
practice round in response to the statement ``As participation in online
experiments increases, society becomes happier.''

\subsection{Results}\label{sec-results-pre}

All analyses were conducted using R (version 4.4.0). We use the
\textbf{irr} package to calculate Fleiss' Kappa to measure interrater
agreement on topic emotionality and strength of correlation for the 25
experimental items. This analysis revealed that participants agreed
above chance for both topic emotionality (\(\kappa\) = 0.07, \emph{p}
\textless{} .001) and strength of correlation (\(\kappa\) = 0.06,
\emph{p} \textless{} .001).

\subsection{Selecting Statements for the Main
Experiment}\label{selecting-statements-for-the-main-experiment}

To control for any potential effects of topic emotionality in the main
experiment, we first select statements that represent neutral emotional
valence. Statements with average topic emotionality ratings between 3
and 5 are statements 2, 10, 22, 16, and 23. To ascertain which
statements represent the greatest consensus, we add standard deviations
in ratings for topic emotionality and strength of correlation. Due to
concerns about experimental power, and in line with evidence that
propensity for belief change is highest when prior beliefs are not
strongly held \citep{xiong_2022}, we elected at this point to test only
the statement corresponding to weak beliefs about the strength of
correlation between the variables in question. We therefore test
statement number 22, ``Higher consumption of spicy foods is associated
with a lower risk of certain types of cancer.'', however we modify the
wording so that both variables (food consumption and cancer risk) are
positively correlated, as previous work indicates that the manipulations
we use in the atypical scatterplot condition are able to change
estimates of correlation in positively correlated scatterplots; no work
regarding the effects of these manipulations in negatively correlated
scatterplots has been completed.

\subsection{Discussion}\label{sec-discussion-pre}

Fleiss' Kappa values for interrater agreement on both topic emotionality
and strength of correlation scales are low (\(\kappa\) = 0.07 and
\(\kappa\) = 0.06 respectively), however do exceed that which would be
expected by chance. We suggest this may be due to Fleiss' Kappa not
being designed with ordinal (Likert scales in this case) data in mind.
In light of this we do not make decisions regarding which statement to
use based on the values of Fleiss' Kappa observed, but rather on the
standard deviations of ratings across all raters. Regardless, we do not
consider this to be a particular weakness, as we also test topic
emotionality and strength of correlation with participants in the main
study and include these ratings as part of our analysis.

\section{Main Study: Potential for Belief Change Using Atypical
Scatterplots}\label{sec-main-study}

We test the statement that exhibited the lowest average level of belief
about correlation, and the 2nd highest level of consensus. Modified for
directionality, this statement is therefore: ``Higher consumption of
plain (non-spicy) foods is associated with a lower risk of certain types
of cancer.''

\subsection{Introduction}\label{introduction}

\subsubsection{Defensive Confidence}\label{defensive-confidence}

In line with evidence that those who are more confident in their ability
to defend their own positions are more susceptible to having those
positions changed \citep{albarracin_2004}, we test participants'
defensive confidence using a 12-item scale. This scale is replicated
from previous work in the supplemental material, and has additionally
been utilized more recently \citep{markant_2023} to explore the
potential for attitude change specifically with regards to correlations
in scatterplots.

\begin{itemize}
\tightlist
\item
  don't forget about reverse scoring items
\end{itemize}

\subsection{Stimuli}\label{stimuli}

\begin{itemize}
\tightlist
\item
  build experiment and pilot to see how many people can do
\item
  then this section can be written
\end{itemize}

\subsection{Method}\label{method}

\subsubsection{Participants}\label{participants}

\subsubsection{Design}\label{design}

We employ a between-participants design. Each participants was randomly
assigned to either group A, in which case they viewed typical
scatterplots, or group B, in which they viewed atypical scatterplots
designed deliberately to elicit higher levels of belief change.

\subsubsection{Procedure}\label{procedure}

\subsection{Results}\label{results}

\subsection{Discussion}\label{discussion}

\section{General Discussion}\label{general-discussion}

\section{Limitations}\label{limitations}

\section{Future Work}\label{future-work}

\section{Conclusion}\label{conclusion}

\bibliographystyle{ACM-Reference-Format}
\bibliography{atypical-scatterplots.bib}

%% begin pandoc before-bib
%% end pandoc before-bib
%% begin pandoc biblio
%% end pandoc biblio
%% begin pandoc include-after
%% end pandoc include-after
%% begin pandoc after-body
%% end pandoc after-body

\end{document}
\endinput
%%
%% End of file `sample-manuscript.tex'.
